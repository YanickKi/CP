\section*{Aufgabe 1}
\subsection*{a)}
Für das Elektrische Feld wurde die 2-Punkt Regel für die Differentiation innerhalb und der standard Differenzenquotient an den Ränder verwendet.
\subsection*{b)}
In Abbildung \ref{fig:1bg} ist das Ergebnis für das Potential und das elektrische Feld des Algorithmus dargestellt.
Dabei wurden die Randbedingungen $\phi = 0$ und die Anfangsbedingungen $\phi(x,y) = 0$ im inneren verwendet. Zudem wurde keine Ladung eingefügt.
Es ist zu erkennen, dass das Potential und das elektrische Feld nicht Perfekt zentriert sind. Dies ist wahrscheinlich eine Folge der geringen Anzahl an Gitterplätzen.
Sonst funktioniert der Algorithmus sehr gut.
\begin{figure}
    \begin{subfigure}{0.48\textwidth}
        \centering
        \includegraphics[height = 5cm]{build/ex_1a.pdf}
        \caption{$\phi(x,y)$}
    \end{subfigure}
    \hfill
    \begin{subfigure}{0.48\textwidth}
        \centering
        \includegraphics[height = 5cm]{build/ex_1b_el.pdf}
        \caption{$\vec{\text{E}}(x,y)$}
    \end{subfigure}
    \caption{Potential und elektrisches Feld für die Randbedingungen aus der b)}
    \label{fig:1bg}
\end{figure}

\FloatBarrier
\subsection*{c)}
In Abbildung \ref{fig:1c} sind die Ergebnisse des Algorithmus dargestellt. Dabei wurde zu den Bedingungen aus der b), die Randbedingung
$\phi(x, y = 1) = 1$ hinzugefügt. 
Die analytische Lösung
\begin{equation}
    \text{\Phi}(x,y) = \sum_{n-1}^\infty \frac{2(1-\cos(n\pi))}{n\pi \sinh(n\pi)} \cdot \sin(n\pi x)\sinh(n\pi y)
\end{equation}
wurde bis $n = 200$ iteriert, da danach zu große Fehler bzw. Unendlichkeitsprobleme auftreten.
\\
Verglichen mit der analytischen Lösung in Abbildung \ref{fig:1ce} fällt auf, dass die Abbildungen sehr ähnlich sind.
Der Auffälligste Unterschied ist, dass im Algorithmus ein sehr starker Sprung an den Ecken Auftritt, während die analytische Lösung dort relativ smooth
übergeht. Dies ist eine Folge der Randbedingungen, welche in dem Algorithmus fest sind, während sie in der analytischen Lösung scheinbar nicht fest sind.

\begin{figure}
    \begin{subfigure}{0.48\textwidth}
        \centering
        \includegraphics[height = 5cm]{build/ex_1c.pdf}
        \caption{Numerische Lösung}
        \label{fig:1c}
    \end{subfigure}
    \hfill
    \begin{subfigure}{0.48\textwidth}
        \centering
        \includegraphics[height = 5cm]{build/ex_1c_an.pdf}
        \caption{Analytische Lösung}
        \label{fig:1ce}
    \end{subfigure}
    \caption{Numerische und analytische Lösung von $\phi$ der Randbedingungen aus der Aufgabe c)}
    \label{fig:1cg}
\end{figure}



\FloatBarrier
\subsection*{d)}
Durch eine hinzugefügte Ladung mit $q = +1$ am Ort $\vec{r} = (0.5, 0.5)$ ergibt sich ein maximum des Potentials an diesem Ort.
Dieses führt zu einem sphärisch abnehmenden Elektrischen Feld, was mit den Erwartungen übereinstimmt. Die beiden größen sind in Abbildung \ref{fig:1dg} dargestellt.
\begin{figure}
    \begin{subfigure}{0.48\textwidth}
        \centering
        \includegraphics[height = 5cm]{build/ex_1d.pdf}
        \caption{$\phi(x,y)$}
        \label{fig:1d}
    \end{subfigure}
    \hfill
    \begin{subfigure}{0.48\textwidth}
        \centering
        \includegraphics[height = 5cm]{build/ex_1d_el.pdf}
        \caption{$\vec{\text{E}}(x,y)$}
        \label{fig:1de}
    \end{subfigure}
    \caption{Potential und elektrisches Feld für die Randbedingungen aus der d)}
    \label{fig:1dg}
\end{figure}
\FloatBarrier
\subsection*{e)}
Die vier eingefügten Ladungen in diesem Aufgabenteil führen zu einem Quadrupol. 
Potential und Feld von diesem sind in Abbildung \ref{fig:1eg} dargestellt.
Das Potential verläuft wie es zu erwarten ist, bei vier Ladungen mit verschiedenen Vorzeichen.

\begin{figure}
    \begin{subfigure}{0.48\textwidth}
        \centering
        \includegraphics[height = 5cm]{build/ex_1e.pdf}
        \caption{$\phi(x,y)$}
        \label{fig:1e}
    \end{subfigure}
    \hfill
    \begin{subfigure}{0.48\textwidth}
        \centering
        \includegraphics[height = 5cm]{build/ex_1e_el.pdf}
        \caption{$\vec{\text{E}}(x,y)$}
        \label{fig:1ee}
    \end{subfigure}
    \caption{Potential und elektrisches Feld für die Randbedingungen aus der e)}
    \label{fig:1eg}
\end{figure}