\section{Aufgabe 2}
Wir haben in jedem Aufgabenteil 1000 Zeitschritte verwendet.
Die Animaitonen sind in dem Ordner \verb|build| und sind den Teilaufgaben dementsprechend benannt.
\subsection*{a)}
Irgendwie ist die Fläche unter dem Graphen nicht konstant, so dass es aussieht, als würde was verloren gehen.
\subsection*{b)}
Hier haben wir Zeitschritte von $0.00049$ (gerade eben stabil) und $0.00051$ genommen.
Man sieht, dass die Lösung mit einem Zeitschritt von $0.00049$ am Anfang stark osillier, sich dann aber 
\enquote{einkriegt} und stabil bleibt. 
Bei einem Zeitschritt von $0.00051$ zeigen sich starke Oszillationen und Divergenzen.
\subsection*{c)}
Die Animationen zu den verschiedenen Anfangsbedingungen $u_1, u_2, u_3$ tragen den selben Index im Namen.
Man sieht, dass die sich die Konfiguration mit den neun Delta-Peaks ihrem Gleichgewichtszustand nähert.
Das Raumintegral haben wir nicht explizit berechnet. Es sollte aber konstant sein über die Zeit, da es die homogene 
Diffusionsgleichung ist, so dass wir keine externe Quelle haben.