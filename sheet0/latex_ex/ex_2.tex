\section*{Aufgabe 2}
Im Allgemeinenen haben wir hier einfach die Subtraktion von fast gleichen Zahlen vermieden.
Die Abweichungen sind in Prozent angegeben.
\subsection*{a)}
\begin{align*}
    f(x) &= \frac{1}{\sqrt{x}} - \frac{1}{\sqrt{x+1}}
    = \frac{\sqrt{x+1} - \sqrt{x}}{\sqrt{x+1}\sqrt{x}}
    = \frac{\sqrt{x+1} - \sqrt{x}}{\sqrt{x+1}\sqrt{x}} \frac{\sqrt{x+1} + \sqrt{x}}{\sqrt{x+1} + \sqrt{x}} \\
    &= \frac{x+1-x}{x \sqrt{x+1} + (x+1)\sqrt{x}}  =
    \frac{1}{x \sqrt{x+1} + (x+1)\sqrt{x}}
\end{align*}
\begin{figure}
    \centering
    \includegraphics[width = \textwidth]{build/ex_2_a.pdf}
    \caption{Plot zum Aufgabenteil a).}
\end{figure}
Irgendwie liegen beide Formen der Funktion bei Null aber die relative Abweichung ist ungleich Null und sieht sehr komisch aus.
\FloatBarrier
\subsubsection*{b)}
\begin{figure}
    \centering
    \includegraphics[width = \textwidth]{build/ex_2_b.pdf}
    \caption{Plot zum Aufgabenteil b).}
\end{figure}
\begin{equation*}
    f(x) = \frac{1- \cos (x)}{\sin (x)} = \frac{2\sin^2(\frac{x}{2})}{\sin(x)}
\end{equation*}
Beide Funktionen liegen bei Null.
Die Abweichung liegt zu beginn bei $\qty{100}{\percent}$, wonach sie oszillatorisch abnimmt.
Es ist zu erkennen, dass sich die Funktion deutlich stabiler nach der Umformung verhält.
\FloatBarrier
\subsubsection*{c)}
Beide Formen liegen wieder konstant bei Null. 
Die Abweichung hat bei ca. $\sfrac{\pi}{2}$ einen sehr hohen Peak.
Wir wundern uns etwas darüber, dass die Funktionen bei Null liegen aber der relative Fehler zwischenzeitlich so riesig wird.
\begin{figure}
    \centering
    \includegraphics[width = \textwidth]{build/ex_2_c.pdf}
    \caption{Plot zum Aufgabenteil c). $\delta = \num{1e-15}$}
    \label{fig:c}
\end{figure}
\begin{align*}
    f(x) &= \sin(x + \delta) - \sin(x)                                                              \\
    &= \sin(x) \cos(\delta) + \cos(x) \sin (\delta) - \sin(x)                                       \\
    &= \sin(x) (\cos(\delta) -1) + \cos(x) \sin(\delta)                                             \\
    &= \cos(x) \sin(\delta) + \sin(x) (\cos(\delta) -1) \frac{(\cos(\delta) +1)}{(\cos(\delta) +1)} \\
    &= \cos(x) \sin(\delta) + \frac{\sin(x) (\cos^2(\delta) - 1)}{(\cos(\delta) +1)}                \\
    &= \cos(x) \sin(\delta) + \frac{\sin(x) \sin^2(\delta)}{(\cos(\delta) +1)}                      \\
\end{align*}  
