\section*{Aufgabe 3}
\subsection*{a)}
In der Abbildung \ref{fig:euler} ist das exakte Ergebnis und der numerische Weg mit dem (symmetrischen)
Euler-Verfahren dargestellt.
Das Zeitintervall $t \in [0, 10]$ wurde einerseits in $N = 150$ und andererseits in $N = \num{1e5}$ Schritte 
unterteilt, so dass Ergebnisse verschiedener Gößenordnungen der Schritte diskutiert werden können.
Bei dem Euler-Verfahren mit dem Startwert $y_0 = 1$ sind kaum Abweichungen zu erkennen, wobei 
bei dem symmetrischen Euler-Verfahren mit den Startwerten $y_0 = 1$ und $y_1 = \symup{e}^{-\symup{\Delta}t}$ vor allen Dingen bei großen Werten von $t$ 
starke Oszillationen zu verzeichen sind.
\begin{figure}
    \centering
    \includegraphics[width = \textwidth]{build/ex_3.pdf}
    \caption{Exakte (Exakt) und mit dem Euler (Euler) - bzw. symmetrischen Euler-Verfahren (Symm.) numerisch ermittelten Funktionswerte für
    $N = 150$ und $N = \num{1e5}$ Schritte. 
    Startwert Euler-Verfahren: $y_0 = 1$. Startwerte symmetrisches Eulerverfahren: $y_0 = 1$ und $y_1 = \symup{e}^{-\symup{\Delta}t}$.}
    \label{fig:euler}
\end{figure} 
\subsection*{b)}
In der Abbildung \ref{fig:euler_b} sind erneut die Ergebnisse wie in Aufgabenteil a) aufgetragen, nur mit dem Unterschied, dass 
die Startwerte geändert wurden.
Dise betragen nun beim Euler-Verfahren $y_0 = 1 - \symup{\Delta}t$ und beim symmetrischen Euler-Verfahren 
$y_0 = 1$ und $y_1 = 1 - \symup{\Delta}t$.
Bei dem Euler-Verfahren lassen sich bei $N=150$ Schritten klar größere Abweichungen als in Aufgabenteil a) erkennen, während bei $N = \num{1e5}$
Schritten keine größeren Abweichungen zu verzeichen sind.
Es zeigen sich deutliche Verbesserungen bei dem symmetrischen Euler-Verfahren. 
Bereits bei $N = 150$ Schritten sind die Ausschläge von knapp 500 auf 20 gesunken, 
während bei $N = \num{1e5}$ Schritten keine höhere Ungenauigkeit als bei dem Euler-Verfahren zu erkennen sind.
\begin{figure}
    \centering
    \includegraphics[width = \textwidth]{build/ex_3_b.pdf}
    \caption{Exakte (Exakt) und mit dem Euler (Euler) - bzw. symmetrischen Euler-Verfahren (Symm.) numerisch ermittelten Funktionswerte für
    $N = 150$ und $N = \num{1e5}$ Schritte.
    Startwert Euler-Verfahren: $y_0 = 1 - \symup{\Delta}t$. 
    Startwerte symmetrisches Eulerverfahren: $y_0 = 1$ und $y_1 = 1 - \symup{\Delta}t$.}
    \label{fig:euler_b}
\end{figure}
Daraus lässt sich schließen, dass die Anfangsbedingungen eine große Auswirkung auf die Stabilität des numerischen Verfahrens haben.
Es könnte damit zusammenhängen, dass der Anfangswert $y_0 = 1 - \symup{\Delta}t$ nicht mehr der analytische Anfangswert, also 
$y_0 = 1$ ist, womit es zur Instabilität kommen kann.