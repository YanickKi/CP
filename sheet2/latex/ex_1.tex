\section*{Aufgabe 1}
\subsection*{a)}
Folgende Ersetzungen haben wir gemacht:
\begin{align*}
    r   & \to r     = a r' \\
    s   & \to s     = a s' \\
    z   & \to z     = a z' \\
    p_L & \to p_L = \frac{\gamma}{a} p_L'
\end{align*}
mit den gestrichelten Größen als dimensionsliose Größen.
Somit nehmen die Formgleichungen folgende Gestalt an
\begin{align*}
    \frac{a\, \symup{d}r'}{a \, \symup{d}s'} &= \frac{\symup{d}r' }{\symup{d}s'} = \cos(\Psi)  \\
    \frac{a\, \symup{d}z'}{a \, \symup{d}s'} &= \frac{\symup{d}z'}{\symup{d}s'} = \sin(\Psi) \\
    \frac{\symup{d}\Psi}{ a \, \symup{d}s'} &=  \frac{1}{a} p_L' - \frac{\rho g a z'}{\gamma} - \frac{\sin(\Psi)}{a r'}.
\end{align*}
Die dritte Formgleichung umgestellt ergibt 
\begin{equation*}
    \frac{\symup{d}\Psi}{\symup{d}s'} = p_L' - \frac{\rho g a^2 z'}{\gamma} - \frac{\sin(\Psi)}{r'} .
\end{equation*}
Nun haben wir $\mu = \rho g a^2$ definiert, was konstant ist. 
Dieses $\mu$ lässt sich nun ebenfalls in Einheiten von der Oberflächenspannung $\gamma$ angeben,
womit die Formgleichungen final als 
\begin{align*}
    \frac{\symup{d}r'}{\symup{d}s'} &= \cos(\Psi)   \\
    \frac{\symup{d}z'}{\symup{d}s'} &= \sin(\Psi)   \\
    \frac{\symup{d}\Psi}{\symup{d}s'} &=  p_L' - \mu' z' - \frac{\sin(\Psi)}{r'}
\end{align*}
geschrieben werden können.
Im Folgenden werden die Striche weggelassen und nur mit den dimensionslosen 
Formgleichungen gerechnet.
\subsection*{b)}
to be continued..
\FloatBarrier
\subsection*{c)}
Hier haben wir eine Schrittweite von $h=\num{1e-4}$ verwendet.
\begin{figure}
    \centering
    \includegraphics[width = 0.95 \textwidth]{build/ex_2c_r.pdf}
\end{figure}
\begin{figure}
    \centering
    \includegraphics[width = 0.95 \textwidth]{build/ex_2c_z.pdf}
\end{figure}
\begin{figure}
    \centering
    \includegraphics[width = 0.95 \textwidth]{build/ex_2c_psi.pdf}
\end{figure}