\documentclass[
  captions=tableheading,  % Tabellenüberschriften
  titlepage=firstiscover, % Titelseite ist Deckblatt
]{scrartcl}

% Paket float verbessern
\usepackage{scrhack}

% Warnung, falls nochmal kompiliert werden muss
\usepackage[aux]{rerunfilecheck}

% unverzichtbare Mathe-Befehle
\usepackage{amsmath}
% viele Mathe-Symbole
\usepackage{amssymb}
% Erweiterungen für amsmath
\usepackage{mathtools}

% Fonteinstellungen
\usepackage{fontspec}
% Latin Modern Fonts werden automatisch geladen
% Alternativ zum Beispiel:
%\setromanfont{Libertinus Serif}
%\setsansfont{Libertinus Sans}
%\setmonofont{Libertinus Mono}

% Wenn man andere Schriftarten gesetzt hat,
% sollte man das Seiten-Layout neu berechnen lassen
\recalctypearea{}

% deutsche Spracheinstellungen
\usepackage[ngerman]{babel}


\usepackage[
  math-style=ISO,    % ┐
  bold-style=ISO,    % │
  sans-style=italic, % │ ISO-Standard folgen
  nabla=upright,     % │
  partial=upright,   % ┘
  warnings-off={           % ┐
    mathtools-colon,       % │ unnötige Warnungen ausschalten
    mathtools-overbracket, % │
  },                       % ┘
]{unicode-math}

% traditionelle Fonts für Mathematik
\setmathfont{Latin Modern Math}
% Alternativ zum Beispiel:
%\setmathfont{Libertinus Math}

\setmathfont{XITS Math}[range={scr, bfscr}]
\setmathfont{XITS Math}[range={cal, bfcal}, StylisticSet=1]

% Zahlen und Einheiten
\usepackage[
  locale=DE,                   % deutsche Einstellungen
  separate-uncertainty=true,   % immer Unsicherheit mit \pm
  per-mode=symbol-or-fraction, % / in inline math, fraction in display math
]{siunitx}

% chemische Formeln
\usepackage[
  version=4,
  math-greek=default, % ┐ mit unicode-math zusammenarbeiten
  text-greek=default, % ┘
]{mhchem}

% richtige Anführungszeichen
\usepackage[autostyle]{csquotes}

% schöne Brüche im Text
\usepackage{xfrac}

% Standardplatzierung für Floats einstellen
\usepackage{float}
\floatplacement{figure}{htbp}
\floatplacement{table}{htbp}

% Floats innerhalb einer Section halten
\usepackage[
  section, % Floats innerhalb der Section halten
  below,   % unterhalb der Section aber auf der selben Seite ist ok
]{placeins}

% Seite drehen für breite Tabellen: landscape Umgebung
\usepackage{pdflscape}

% Captions schöner machen.
\usepackage[
  labelfont=bf,        % Tabelle x: Abbildung y: ist jetzt fett
  font=small,          % Schrift etwas kleiner als Dokument
  width=0.9\textwidth, % maximale Breite einer Caption schmaler
]{caption}
% subfigure, subtable, subref
\usepackage{subcaption}

% Grafiken können eingebunden werden
\usepackage{graphicx}

% schöne Tabellen
\usepackage{booktabs}

% Verbesserungen am Schriftbild
\usepackage{microtype}


% Hyperlinks im Dokument
\usepackage[
  german,
  unicode,        % Unicode in PDF-Attributen erlauben
  pdfusetitle,    % Titel, Autoren und Datum als PDF-Attribute
  pdfcreator={},  % ┐ PDF-Attribute säubern
  pdfproducer={}, % ┘
]{hyperref}
% erweiterte Bookmarks im PDF
\usepackage{bookmark}

% Trennung von Wörtern mit Strichen
\usepackage[shortcuts]{extdash}

\author{%
  Yanick Kind\\%
  \and%
  Till Willershausen\\%
  \and%
  Justus Dung\\%
}

\title{Blatt 0}

\begin{document}

\maketitle
\thispagestyle{empty}
\newpage

\section*{Aufgabe 1}
\subsubsection*{a)}
\begin{figure}
    \centering
    \includegraphics[width = 0.85 \textwidth]{build/ex_1a_varstep.pdf}
    \caption{Ableitung von $\sin(x)$ an der Stelle $x=0$ ermittelt mit der Zweipunktsregel
    in Abhängigkeit der Schrittweite $h$.}
    \label{fig:h_dep}
\end{figure}

\begin{figure}
    \centering
    \includegraphics[width = 0.85 \textwidth]{build/ex_1a_compare.pdf}
    \caption{Ableitung des Sinus analytisch und numerisch ermittelt in Abhängigkeit von 
    $x$ mit einer festen Schrittweite $h=0.3$.}
    \label{fig:a_comp}
\end{figure}

\begin{figure}
    \centering
    \includegraphics[width = 0.85 \textwidth]{build/ex_1a_error.pdf}
    \caption{Relative Abweichung der numerischen von den analytischen Werten der Ableitung von $\sin(x)$
    bei einer Schrittweite von $h=0.3$.}
    \label{fig:rel_a}
\end{figure}
\FloatBarrier
%
%\subsection*{b)}
%\begin{figure}
%    \centering
%    \includegraphics[width = 0.85 \textwidth]{build/ex_1b_compare.pdf}
%    \caption{Zweite Ableitung des Sinus analytisch und numerisch ermittelt in Abhängigkeit von 
%    $x$ mit einer festen Schrittweite $h=0.3$.}
%    \label{fig:b_comp}
%\end{figure}
%
%\begin{figure}
%    \centering
%    \includegraphics[width = 0.85 \textwidth]{build/ex_1b_error.pdf}
%    \caption{Relative Abweichung der numerischen von den analytischen Werten der zweiten Ableitung von $\sin(x)$
%    bei einer Schrittweite von $h=0.3$.}
%    \label{fig:rel_b}
%\end{figure}
%
%\begin{equation*}
%    f^{\prime \prime} (x) = \frac{f^{\prime}(x+h) - f^{\prime}(x-h)}{2h}    
%\end{equation*}
%Mit der Zweipunktsregel einfach wieder einsetzen.
%
%\begin{align*}
%    f^{\prime \prime} (x) = \frac{f(x+2h) - 2 f(x) + f(x-2h)}{4h^2}
%\end{align*}
%\FloatBarrier
%\subsection*{c)}
%
%\begin{figure}
%    \centering
%    \includegraphics[width = 0.85 \textwidth]{build/ex_1c_compare.pdf}
%    \caption{Zweite Ableitung des Sinus analytisch und numerisch ermittelt in Abhängigkeit von 
%    $x$ mit einer festen Schrittweite $h=0.3$ (Vierpunktsregel).}
%    \label{fig:c_comp}
%\end{figure}
%
%\begin{figure}
%    \centering
%    \includegraphics[width = 0.85 \textwidth]{build/ex_1c_error.pdf}
%    \caption{Relative Abweichung der numerischen von den analytischen Werten der ersten Ableitung von $\sin(x)$
%    bei einer Schrittweite von $h=0.3$ (Vierpunktsregel).}
%    \label{fig:rel_c}
%\end{figure}
%
\section{Aufgabe 2}
\subsection*{a)}
Als erstes muss man die Varibalen diskretisieren:
\begin{equation*}
    u(x,y,t) \to u(i \symup{\Delta}x, j\symup{\Delta}y, n \symup{\Delta}t) = u_{i,j}^n
\end{equation*}
mit den Schrittweiten $\symup{\Delta}x, \symup{\Delta}y, \symup{\Delta}t$ in $x$-, $y$- Richtung und in der Zeit.
Nun nutze einfach die Approximation der zweiten Ableitung.
\begin{align*}
    \partial_x u_{x,y,t} &= \frac{u_{i+1,j}^n -2 u_{i,j}^n + u_{i-1,j}^n }       {\symup{\Delta}x^2}\\
    \partial_y u_{x,y,t} &= \frac{u_{i,j+1}^n -2 u_{i,j}^n + u_{i,j-1}^n }       {\symup{\Delta}y^2}\\
    \partial_t u_{x,y,t} &= \frac{u_{i,j}^{n+1} -2 u_{i,j}^{n} + u_{i,j}^{n-1} }   {\symup{\Delta}t^2}
\end{align*}
Die partiellen Ableitungen werden dann in die zweidimensionale Wellengleichung eingesetzt, so dass sich 
\begin{align*}
    \frac{u_{ij}^{n+1} -2 u_{i,j}^{n} + u_{i,j}^{n-1} }   {\symup{\Delta}t^2} &= c^2
    \left ( \frac{u_{i+1,j}^n -2 u_{i,j}^n + u_{i-1,j}^n } {\symup{\Delta}x^2} + \frac{u_{i,j+1}^n -2 u_{i,j}^n + u_{i,j-1}^n } {\symup{\Delta}y^2} \right ) \\
    \iff u_{ij}^{n+1} -2 u_{i,j}^{n} + u_{i,j}^{n-1} &= 
    c^2 \symup{\Delta} t^2 \left ( \frac{u_{i+1,j}^n -2 u_{i,j}^n + u_{i-1,j}^n } {\symup{\Delta}x^2} + \frac{u_{i,j+1}^n -2 u_{i,j}^n + u_{i,j-1}^n } {\symup{\Delta}y^2} \right )
\end{align*}
ergibt.
\subsection*{c)}
Aus der Bedingung $\partial_t u = 0$ folgt 
\begin{equation}
    \frac{u_{i,j}^1 - u_{i,j}^{-1}}{2\symup{\Delta}t } = 0 \iff u_{i,j}^1 = u_{i,j}^{-1}. \label{eqn:dt0}
\end{equation}
Hier wurde einfach die Ableitung bei dem Index $n=0$ benutzt.
Hier wird jedoch noch $u_{i,j}^{-1}$ benötigt.
Dazu wird in der Wellengleichung $n=0$ gesetzt, so dass 
\begin{align*}
    u_{ij}^{1} -2 u_{i,j}^{0} + u_{i,j}^{-1} &= c^2 \symup{\Delta} t^2 
    \left ( \frac{u_{i+1,j}^0 -2 u_{i,j}^0 + u_{i-1,j}^0 } {\symup{\Delta}x^2} 
    + \frac{u_{i,j+1}^0 -2 u_{i,j}^0 + u_{i,j-1}^0 } {\symup{\Delta}y^2} \right ) \\
    \iff u_{i,j}^-1 &= c^2 \symup{\Delta} t^2 
    \left ( \frac{u_{i+1,j}^0 -2 u_{i,j}^0 + u_{i-1,j}^0 } {\symup{\Delta}x^2} 
    + \frac{u_{i,j+1}^0 -2 u_{i,j}^0 + u_{i,j-1}^0 } {\symup{\Delta}y^2} \right ) + 2 u_{i,j}^0 - u_{i,j}^1
\end{align*}
folgt.
Dieses Resultat wird in Gleichung \eqref{eqn:dt0} eingesetzt, so dass 
\begin{equation*}
    u_{i,j}^1 = \frac{c^2 \symup{\Delta} t^2}{2}
    \left ( \frac{u_{i+1,j}^0 -2 u_{i,j}^0 + u_{i-1,j}^0 } {\symup{\Delta}x^2} 
    + \frac{u_{i,j+1}^0 -2 u_{i,j}^0 + u_{i,j-1}^0 } {\symup{\Delta}y^2} \right ) + u_{i,j}^0
\end{equation*}
folgt.
Da $u_{i,j}^0$ für alle $i,j$ bekannt ist, lässt sich $u_{i,j}^1$ so ermitteln.


\end{document}
