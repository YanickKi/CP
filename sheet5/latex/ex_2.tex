\section{Aufgabe 2}
\subsection*{a)}
Als erstes muss man die Varibalen diskretisieren:
\begin{equation*}
    u(x,y,t) \to u(i \symup{\Delta}x, j\symup{\Delta}y, n \symup{\Delta}t) = u_{i,j}^n
\end{equation*}
mit den Schrittweiten $\symup{\Delta}x, \symup{\Delta}y, \symup{\Delta}t$ in $x$-, $y$- Richtung und in der Zeit.
Nun nutze einfach die Approximation der zweiten Ableitung.
\begin{align*}
    \partial_x u_{x,y,t} &= \frac{u_{i+1,j}^n -2 u_{i,j}^n + u_{i-1,j}^n }       {\symup{\Delta}x^2}\\
    \partial_y u_{x,y,t} &= \frac{u_{i,j+1}^n -2 u_{i,j}^n + u_{i,j-1}^n }       {\symup{\Delta}y^2}\\
    \partial_t u_{x,y,t} &= \frac{u_{i,j}^{n+1} -2 u_{i,j}^{n} + u_{i,j}^{n-1} }   {\symup{\Delta}t^2}
\end{align*}
Die partiellen Ableitungen werden dann in die zweidimensionale Wellengleichung eingesetzt, so dass sich 
\begin{align*}
    \frac{u_{ij}^{n+1} -2 u_{i,j}^{n} + u_{i,j}^{n-1} }   {\symup{\Delta}t^2} &= c^2
    \left ( \frac{u_{i+1,j}^n -2 u_{i,j}^n + u_{i-1,j}^n } {\symup{\Delta}x^2} + \frac{u_{i,j+1}^n -2 u_{i,j}^n + u_{i,j-1}^n } {\symup{\Delta}y^2} \right ) \\
    \iff u_{ij}^{n+1} -2 u_{i,j}^{n} + u_{i,j}^{n-1} &= 
    c^2 \symup{\Delta} t^2 \left ( \frac{u_{i+1,j}^n -2 u_{i,j}^n + u_{i-1,j}^n } {\symup{\Delta}x^2} + \frac{u_{i,j+1}^n -2 u_{i,j}^n + u_{i,j-1}^n } {\symup{\Delta}y^2} \right )
\end{align*}
ergibt.
\subsection*{c)}
Aus der Bedingung $\partial_t u\bigr |_{t=0} = 0$ folgt mit der symmetrischen Zweipunktsformel 
\begin{equation}
    \frac{u_{i,j}^1 - u_{i,j}^{-1}}{2\symup{\Delta}t } = 0 \iff u_{i,j}^1 = u_{i,j}^{-1}. \label{eqn:dt0}
\end{equation}
Hier wurde einfach die Ableitung bei dem Index $n=0$ benutzt.
Hier wird jedoch noch $u_{i,j}^{-1}$ benötigt.
Dazu wird in der Wellengleichung $n=0$ gesetzt, so dass 
\begin{align*}
    u_{ij}^{1} -2 u_{i,j}^{0} + u_{i,j}^{-1} &= c^2 \symup{\Delta} t^2 
    \left ( \frac{u_{i+1,j}^0 -2 u_{i,j}^0 + u_{i-1,j}^0 } {\symup{\Delta}x^2} 
    + \frac{u_{i,j+1}^0 -2 u_{i,j}^0 + u_{i,j-1}^0 } {\symup{\Delta}y^2} \right ) \\
    \iff u_{i,j}^{-1} &= c^2 \symup{\Delta} t^2 
    \left ( \frac{u_{i+1,j}^0 -2 u_{i,j}^0 + u_{i-1,j}^0 } {\symup{\Delta}x^2} 
    + \frac{u_{i,j+1}^0 -2 u_{i,j}^0 + u_{i,j-1}^0 } {\symup{\Delta}y^2} \right ) + 2 u_{i,j}^0 - u_{i,j}^1
\end{align*}
folgt.
Dieses Resultat wird in Gleichung \eqref{eqn:dt0} eingesetzt, so dass 
\begin{equation*}
    u_{i,j}^1 = \frac{c^2 \symup{\Delta} t^2}{2}
    \left ( \frac{u_{i+1,j}^0 -2 u_{i,j}^0 + u_{i-1,j}^0 } {\symup{\Delta}x^2} 
    + \frac{u_{i,j+1}^0 -2 u_{i,j}^0 + u_{i,j-1}^0 } {\symup{\Delta}y^2} \right ) + u_{i,j}^0
\end{equation*}
folgt.
Da $u_{i,j}^0$ für alle $i,j$ bekannt ist, lässt sich $u_{i,j}^1$ so ermitteln.
\subsection*{d)}
Um die Wellengleichung dimemensionslos zu machen, wird zunächst die Zeit mit einem beliebigen $t_0$ reskaliert
\begin{equation*}
    t \to t' = \frac{t}{t_0} \iff t = t_0 t'.
\end{equation*}
Die Ortsvariablen lassen sich demnach zu 
\begin{equation*}
    x \to x' = \frac{x}{c t_0} \iff x = t_0 c x' \quad y \to y' = \frac{y}{c t_0} \iff y = t_0 c y'
\end{equation*}
umdefinieren.
Diese Koordinatentransformation in die Wellengleichung eingesetzt, liefert 
\begin{align*}
    \frac{\partial^2 u }{t_0^2 \partial t'^2} &= c^2 \left ( \frac{\partial^2 u }{t_0^2 c^2 \partial x'^2} + \frac{\partial^2 u }{t_0^2 c^2  \partial y'^2}\right ) \\
    \frac{\partial^2 u }{\partial t'^2} &= \left ( \frac{\partial^2 u }{ \partial x'^2} + \frac{\partial^2 u }{\partial y'^2}\right ).
\end{align*}
\subsection*{e)}
Hier haben wir eine Zeitschrittweite von $\symup{\Delta} t = \num{1e-5}$ gewählt.
\begin{figure}
    \begin{subfigure}{0.48 \textwidth}
        \centering
        \includegraphics[height = 5cm]{build/2e_0.pdf}
        \caption{Deflection-field bei $t = 0$ (Initialfeld).}
    \end{subfigure}
    \hfill
    \begin{subfigure}{0.48 \textwidth}  
        \centering  
        \includegraphics[height = 5cm]{build/2e_6.pdf}
        \caption{Deflection-field bei $t = 6 \symup{\Delta}t$ also nach 6 Zeitschritten.}
    \end{subfigure}
\end{figure}