\section*{Aufgabe 1}
\subsubsection*{a)}
\begin{figure}
    \centering
    \includegraphics[width = 0.85 \textwidth]{build/ex_1a_varstep.pdf}
    \caption{Ableitung von $\sin(x)$ an der Stelle $x=0$ ermittelt mit der Zweipunktsregel
    in Abhängigkeit der Schrittweite $h$.}
    \label{fig:h_dep}
\end{figure}

\begin{figure}
    \centering
    \includegraphics[width = 0.85 \textwidth]{build/ex_1a_compare.pdf}
    \caption{Ableitung des Sinus analytisch und numerisch ermittelt in Abhängigkeit von 
    $x$ mit einer festen Schrittweite $h=0.3$.}
    \label{fig:a_comp}
\end{figure}

\begin{figure}
    \centering
    \includegraphics[width = 0.85 \textwidth]{build/ex_1a_error.pdf}
    \caption{Relative Abweichung der numerischen von den analytischen Werten der Ableitung von $\sin(x)$
    bei einer Schrittweite von $h=0.3$.}
    \label{fig:rel_a}
\end{figure}
\FloatBarrier

\subsection*{b)}
\begin{figure}
    \centering
    \includegraphics[width = 0.85 \textwidth]{build/ex_1b_compare.pdf}
    \caption{Zweite Ableitung des Sinus analytisch und numerisch ermittelt in Abhängigkeit von 
    $x$ mit einer festen Schrittweite $h=0.3$.}
    \label{fig:b_comp}
\end{figure}

\begin{figure}
    \centering
    \includegraphics[width = 0.85 \textwidth]{build/ex_1b_error.pdf}
    \caption{Relative Abweichung der numerischen von den analytischen Werten der zweiten Ableitung von $\sin(x)$
    bei einer Schrittweite von $h=0.3$.}
    \label{fig:rel_b}
\end{figure}

\begin{equation*}
    f^{\prime \prime} (x) = \frac{f^{\prime}(x+h) - f^{\prime}(x-h)}{2h}    
\end{equation*}
Mit der Zweipunktsregel einfach wieder einsetzen.

\begin{align*}
    f^{\prime \prime} (x) = \frac{f(x+2h) - 2 f(x) + f(x-2h)}{4h^2}
\end{align*}

\FloatBarrier