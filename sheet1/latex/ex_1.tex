\section*{Aufgabe 1}
\subsubsection*{a)}
\begin{figure}
    \centering
    \includegraphics[width = 0.85 \textwidth]{build/ex_1a_varstep.pdf}
    \caption{Ableitung von $f_1(x)$ an der Stelle $x=0$ ermittelt mit der Zweipunktsregel
    in Abhängigkeit der Schrittweite $h$.}
    \label{fig:h_dep}
\end{figure}
\noindent
In Abbildung \ref{fig:h_dep} ist klar zu erkennen, dass sich der Wert der Ableitung bei $x=0$
von der Funktion $f_1(x) = \sin(x)$ für kleine
Schrittweiten der eins annähert (und somit genauer wird).
Jedoch ist zu beachten, dass bei der Zwei- und Vierpunktsregel bei kleineren 
Schrittweiten die Auslöschung dominanter wird, womit in den Daten ein Rauschen zustande kommt 
und deswegen ein Kompormiss gefunden werden muss. 
Im Folgenden wird bei allen Teilaufgaben eine Schrittweite von $h = \num{1e-2}$ gewählt.
\begin{figure}
    \centering
    \includegraphics[width = 0.85 \textwidth]{build/ex_1a_compare.pdf}
    \caption{Ableitung von $f_1(x)$ analytisch und numerisch ermittelt in Abhängigkeit von 
    $x$ mit einer festen Schrittweite $h=\num{1e-2}$.}
    \label{fig:a_comp}
\end{figure}
Es lässt sich in Abbildung \ref{fig:a_comp} erkennen, dass die 
numerische Berechnung der ersten Ableitung von $f_1(x)$ nah an den 
analytischen Werten liegt. 
Jedoch sind in Abbildung \ref{fig:rel_a} bei dem relativen Fehler 
$\symup{\Delta} f^{\prime}_1(x)$ der numersichen im Bezug 
auf die analytischen Werte hohe Peaks und stärkeres Rauschen um $\sfrac{-\pi}{2}$ und $\sfrac{\pi}{2}$
zu verzeichnen. 
Dies könnte daran liegen, dass bei dem relativen Fehler durch 
die analytische Ableitung geteilt wird und an diesen Stellen 
die Ableitung von $f_1(x)$, $\cos(x)$, die Nullstellen liegen und 
somit Pole auftreten.
\begin{figure}
    \centering
    \includegraphics[width = 0.85 \textwidth]{build/ex_1a_error.pdf}
    \caption{Relative Abweichung der numerischen von den analytischen Werten der Ableitung von $f_1(x)$
    bei einer Schrittweite von $h=\num{1e-2}$.}
    \label{fig:rel_a}
\end{figure}
\FloatBarrier
\subsection*{b)}
Die Zweite Ableitung lässt sich mittels der Zweipunktsregel durch 
\begin{equation}
    f^{\prime \prime} (x) = \frac{f^{\prime}(x+h) - f^{\prime}(x-h)}{2h} \label{eqn:twotwo}
\end{equation}
bestimmen.
Dazu wird die Ableitung an den Stellen $x+h$ und $x-h$ benötigt.
Die erste Ableitung wird an den Stellen ebenfalls mit der 
Zweipunktsregel bestimmt und in Glechung \eqref{eqn:twotwo} eingesetzt,
womit
\begin{align*}
    f^{\prime \prime} (x) = \frac{f(x+2h) - 2 f(x) + f(x-2h)}{4h^2}
\end{align*}
folgt.
\begin{figure}
    \centering
    \includegraphics[width = 0.85 \textwidth]{build/ex_1b_compare.pdf}
    \caption{Zweite Ableitung von $f_1(x)$ analytisch und numerisch ermittelt in Abhängigkeit von 
    $x$ mit einer festen Schrittweite $h=\num{1e-2}$.}
    \label{fig:b_comp}
\end{figure}
Erneut lässt sich eine hohe Genauigkeit der numerischen Berechnung 
der zweiten Ableitung von $f_1(x)$, $- \sin(x)$, beobachten.
Bei dem relativen Fehler $\symup{\Delta} f^{\prime}_1(x)$  ist zwar erneut Rauschen zu sehen, jedoch ist dieses 
homogener über den Definitionsbereich verteilt. 
Ebenso ist nur ein Peak um $-\pi$ zu erkennen.
\begin{figure}
    \centering
    \includegraphics[width = 0.85 \textwidth]{build/ex_1b_error.pdf}
    \caption{Relative Abweichung der numerischen von den analytischen Werten der zweiten Ableitung von $f_1(x)$
    bei einer Schrittweite von $h=\num{1e-2}$.}
    \label{fig:rel_b}
\end{figure}
\FloatBarrier
\subsection*{c)}
Erneut liegen die mit der Vierpunktsregel ermittelten Werte nah an den analytischen Werten.
Ebenfalls lassen sich bei der relativen Abweichung (s. Abbildung \ref{fig:rel_c}) wieder Peaks, 
welche bei $-\sfrac{\pi}{2}$ und $\sfrac{\pi}{2}$ liegen, und
Rauschen, besonders an den Peaks, verzeichnen.

\begin{figure}
    \centering
    \includegraphics[width = 0.85 \textwidth]{build/ex_1c_compare.pdf}
    \caption{Erste Ableitung von $f_1(x)$ analytisch und numerisch ermittelt in Abhängigkeit von 
    $x$ mit einer festen Schrittweite $h=\num{1e-2}$ (Vierpunktsregel).}
    \label{fig:c_comp}
\end{figure}

\begin{figure}
    \centering
    \includegraphics[width = 0.85 \textwidth]{build/ex_1c_error.pdf}
    \caption{Relative Abweichung der numerischen von den analytischen Werten der ersten Ableitung von $f_1(x)$
    bei einer Schrittweite von $h=\num{1e-2}$ (Vierpunktsregel).}
    \label{fig:rel_c}
\end{figure}
\FloatBarrier
\subsubsection*{d)}
Wie in Abbildung \ref{fig:d_comp} zu sehen ist, liegen 
die Werte der ersten Ableitung von der Funktion 
\begin{equation*}
    f_2(x)
    \begin{cases}
        & 2 \lfloor \frac{x}{\pi} \rfloor - \cos(x \text{ mod } \pi) + 1 \quad \text{für } x \geq 0 \\
        & 2 \lfloor \frac{x}{\pi} \rfloor - \cos(x \text{ mod } \pi) + 1 \quad \text{für } x < 0 \\
    \end{cases}
\end{equation*}
\begin{figure}
    \centering
    \includegraphics[width = 0.85 \textwidth]{build/ex_1d_compare.pdf}
    \caption{Erste Ableitung von $f_2(x)$ mit der Zwei- und Vierpunktsregel ermittelt mit einer festen von Schrittweite $h=\num{1e-2}$.}
    \label{fig:d_comp}
\end{figure}
\begin{figure}
    \centering
    \includegraphics[width = 0.85 \textwidth]{build/ex_1d_error.pdf}
    \caption{Relative Abweichung der ersten Ableitung von $f_2(x)$ ermittelt mit der Vierpunktsregel von 
    den Werten, welche mit der Zweipunktsregel berechnet worden sind.
    Die Schrittweite beträgt $h=\num{1e-2}$.}
    \label{fig:rel_d}
\end{figure}
Der relative Fehler $\symup{\Delta} f^{\prime}_{2, \text{Vierpunkt}}(x)$
zeigt sich besonders an den Rändern, welcher dort exponentiell wächst.
Um den Peak bei $x=0$ ist der relative Fehler minimal.